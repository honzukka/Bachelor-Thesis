\chapter{Attachments}

%%-----------------------------------------------------------------------------------------
%% SECTION
%%-----------------------------------------------------------------------------------------
\section{Parameter Description}
\label{sec:parameter_description}

As mentioned in section \ref{sec:implementation_algorithm}, the proposed visualizations can be configured by a set of clustering parameters, a set of visualization parameters and a few other parameters. We will now provide an overview of all of them.

%%-----------------------------------------------------------------------------------------
\subsection{Clustering Parameters}
\label{sec:clustering_parameters}

\begin{description}
\item [Cluster Count] Determines the number of clusters to be retrieved from the dendrogram (see Fig. \ref{fig:telea-hierarchical_clustering}) and used for visualization. If the dendrogram is a tree, any valid number of clusters is guaranteed to cover the whole data set. If the dendrogram is a forest (see section \ref{sec:analysis_clustering_algorithm}), certain parts of the data set may remain uncovered by the chosen clusters. {\bf Valid values:} \(\bm{[1,S]}\) where \(S\) is the number of vertices of one of the {\it scenes}.
\item [Direction Significance] Determines how large the direction weight coefficient in the error function (see Eq. \ref{eq:clustering_error}) will be. See section \ref{sec:parameter_effect} for details. It is only meaningful in combination with all the other significance parameters\footnotemark. {\bf Valid values:} \(\bm{[0,100]}\)
\item [Magnitude Significance] Determines how large the magnitude weight coefficient in the error function (see Eq. \ref{eq:clustering_error}) will be. See section \ref{sec:parameter_effect} for details. It is only meaningful in combination with all the other significance parameters\footnotemark. {\bf Valid values:} \(\bm{[0,100]}\)
\item [Position Significance] Determines how large the position weight coefficient in the error function (see Eq. \ref{eq:clustering_error}) will be. See section \ref{sec:parameter_effect} for details. It is only meaningful in combination with all the other significance parameters\footnotemark. {\bf Valid values:} \(\bm{[0,100]}\)
\item [Resolution Significance] Determines how large the mesh resolution weight coefficient in the error function (see Eq. \ref{eq:clustering_error}) will be. See section \ref{sec:parameter_effect} for details. It is only meaningful in combination with all the other significance parameters\footnotemark. {\bf Valid values:} \(\bm{[0,100]}\)
\end{description}

\addtocounter{footnote}{-4}
\stepcounter{footnote}\footnotetext{\label{conversion}Here is the conversion between {\it significance} and {\it weight}: Consider {\it significance} values \(s_d,s_m,s_p,s_r \in [1,100]\) and {\it weight} values \(k_d,k_m,k_p,k_r \in [0,1]\). We need \(k_d + k_m + k_p + k_r = 1\), therefore \(k_d = s_d / {(s_d + s_m + s_p + s_r)}\) and similarly for all other weights.}
\stepcounter{footnote}\footnotetext{See footnote \ref{conversion}.}
\stepcounter{footnote}\footnotetext{See footnote \ref{conversion}.}
\stepcounter{footnote}\footnotetext{See footnote \ref{conversion}.}

%%-----------------------------------------------------------------------------------------
\subsection{Visualization Parameters}
\label{sec:visualization_parameters}

\begin{description}
\item [Arrow Height Minimum Scale] Determines the height of an arrow representing the lowest metric value present in the data set. This value will multiply the height of the default arrow (see section \ref{sec:implementation_visualizers}) and therefore does not represent the absolute value of the minimum arrow height. {\bf Valid values:} \(\bm{[0.1,10]}\)

\item [Arrow Height Maximum Scale] Determines the height of an arrow representing the highest metric value present in the data set. This value will multiply the height of the default arrow (see section \ref{sec:implementation_visualizers}) and therefore does not represent the absolute value of the maximum arrow height. {\bf Valid values:} \(\bm{[0.1,10]}\)

\item [Arrow Width Minimum Scale] Determines the width of an arrow representing a cluster with the smallest area out of all visualized clusters. This value will multiply the width of the default arrow (see section \ref{sec:implementation_visualizers}) and therefore does not represent the absolute value of the minimum arrow width. {\bf Valid values:} \(\bm{[0.1,10]}\)

\item [Arrow Width Maximum Scale] Determines the width of an arrow representing a cluster covering the whole data set. This value will multiply the width of the default arrow (see section \ref{sec:implementation_visualizers}) and therefore does not represent the absolute value of the maximum arrow width. {\bf Valid values:} \(\bm{[0.1,10]}\)

\item [Arrow Outwards Color] Determines the color of arrows pointing {\it outwards} (see section \ref{sec:analysis}). {\bf Valid values: Any RGB color}.

\item [Arrow Inwards Color] Determines the color of arrows pointing {\it inwards} (see section \ref{sec:analysis}). {\bf Valid values: Any RGB color}.

\item [Color Metric Outwards] Determines the hue of vertices which are assigned a metric vector pointing {\it outwards} (see sections \ref{sec:analysis} and \ref{sec:analysis-color}). {\bf Valid values: Any RGB color}.

\item [Color Metric Inwards] Determines the hue of vertices which are assigned a metric vector pointing {\it inwards} (see sections \ref{sec:analysis} and \ref{sec:analysis-color}). {\bf Valid values: Any RGB color}.

\item [Color Diff Threshold] In absolute mode (see section \ref{sec:analysis-color})), all vertices with an associated metric vector longer than this value will receive the brightest color. {\bf Valid values:} \(\bm{[1, \infty)}\) (In the MeshDiff UI this is limited by the diameter of the whole scene.)

\item [Disabled Color] The color of vertices which were excluded from the visualization by thresholding. {\bf Valid values: Any RGB color}.

\item [Disabled Threshold Length] All vertices with a shorter associated metric vector will be excluded from the visualization. {\bf Valid values:} \(\bm{[1, \infty)}\) (In the meshDiff UI this is limited by the diameter of the whole scene.)

\item [Disabled Threshold Size] All vertices which are part of a cluster whose area is smaller than this value will be excluded from visualization. {\bf Valid values:} \(\bm{[1, \infty)}\) (In the meshDiff UI this is limited by the diameter of the whole scene squared.)
\end{description}

%%-----------------------------------------------------------------------------------------
\subsection{Other Parameters}

The rest of the parameters comprise the viewing angle and zoom of both {\it scenes} and also the metric type, clustering type and visualization type. They are used to configure the mesh viewer in MeshDiff and they also determine how the \verb+DiffVector+ class is initialized (see section \ref{sec:implementation-framework}).

\begin{description}
\item [Zoom Left \& Right] Determines the zoom factor of the mesh view. {\bf Valid values: Not limited.} (The default value is 1.)
\item [Rotation Left \& Right] Determines the viewing angle of a mesh. {\bf Valid values: Any 4x4 rotation matrix.}
\item [Metric] Determines the metric type used in difference computation. (See section \ref{sec:analysis}.) {\bf Valid values: \{Distance, NormalProjectedDistance\}}
\item [Clustering Type] Determines the type of clustering used in the visualization process. (See section \ref{sec:implementation_clustering}.) {\bf Valid values: \{None, Simple, Signed\}}
\item [Color Visualization] Determines the type of color visualization used. (See section \ref{sec:analysis-color}.) {\bf Valid values: \{None, VertexRelative, VertexAbsolute, ClusterRandom, ClusterRelative, ClusterAbsolute\}}
\item [Arrow Visualization] Determines whether arrow visualization is used. {\bf Valid values: \{True, False\}}
\end{description}

%%-----------------------------------------------------------------------------------------
%% SECTION
%%-----------------------------------------------------------------------------------------
\section{Parameter Loading and Storing}
\label{sec:parameter_load_store}

In this section, we describe how MeshDiff can write its configuration to a file and also read it from a file. First, we will talk about the file format. It is a very simple custom format designed to store key/value pairs and to allow for easy reading and writing. Then we will mention how the process of reading and writing is realized in the code.

%%-----------------------------------------------------------------------------------------
\subsection{Format}

Each file is divided into sections which are delimited by a special line containing the name of the section. What follows are key/value pairs, one pair per line. Here is an example of a full configuration file:

\begin{code}
---Trackball Left---
zoom=1
rotation=1,0,0,0;0,1,0,0;0,0,1,0;0,0,0,1

---Trackball Right---
zoom=1
rotation=1,0,0,0;0,1,0,0;0,0,1,0;0,0,0,1

---Metric---
metric=distance

---Clustering Parameters---
clusterCount=20
directionSignificance=15
positionSignificance=25
magnitudeSignificance=10
resolutionSignificance=0

---Visualizer Parameters---
arrowColorNear=0.5019608,0.5019608,0
arrowColorFar=0.5019608,0.5019608,0
colorNear=0.5019608,0,0.5019608
colorFar=0.5019608,0,0.5019608
disabledColor=1,0.5019608,0
arrowWidthMinScale=10
arrowWidthMaxScale=10
arrowHeightMinScale=10
arrowHeightMaxScale=10
disabledThresholdLength=0.3
disabledThresholdSize=0.8
colorDiffThreshold=0.1

---Form Settings---
clusteringType=None
colorVisualization=None
arrowVisualization=False
\end{code}

All lines which do not fulfill any of the following requirements are ignored:

\begin{itemize}
\item Start with \verb+---+
\item Yield exactly two strings upon split at \verb+=+
\end{itemize}

The format of keys and values is defined by the program.
%%-----------------------------------------------------------------------------------------
\subsection{Loading and Storing}

We have written \verb+ParameterWriter+ and \verb+ParameterReader+ classes which can be used in a similar way other I/O classes are used in C\#.

\verb+ParameterReader+ is initialized with a file path and a section name which is the full delimiting string shown above. Then a \verb+ReadPair()+ method can be called in a loop and key/value pairs are returned as tuples. When there are no more pairs in the given section, \verb+null+ is returned. All objects of this class should be disposed of at the end of their lifetime.

\verb+ParameterWriter+ is also initialized with a file path and a section name and provides two methods: \verb+WritePair()+ and \verb+WriteEmptyLine()+. Both methods write to the given file and they always append to file. Upon the first call, a section of the given name is created and all subsequent calls write key/value pairs under that section. \verb+WritePair()+ can only accept arguments of type \verb+string+. This class also implements the \verb+IDisposable+ interface.