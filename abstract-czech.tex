\documentclass[12pt,a4paper,hyperfootnotes=false]{report}
\setlength\textwidth{145mm}
\setlength\textheight{247mm}
\setlength\oddsidemargin{15mm}
\setlength\evensidemargin{15mm}
\setlength\topmargin{0mm}
\setlength\headsep{0mm}
\setlength\headheight{0mm}
% \openright makes the following text appear on a right-hand page
\let\openright=\clearpage

%% Generate PDF/A-2u
\usepackage[a-2u]{pdfx}

%% Character encoding: usually latin2, cp1250 or utf8:
\usepackage[utf8]{inputenc}

\begin{document}
	
\noindent
Vizualizace rozdílu dvou trojúhelníkových sítí se používá v geometrické morfometrii, která zkoumá tvary biologických objektů, jako jsou kosti, obličejové symetrie a další. Existující vizualizace většinou kódují rozdílové metriky do barvy vrcholu. Tato jednodimenzionální informace ale není dostatečná pro zobrazení většího počtu rozdílových metrik zároveň. Abychom překonali toto omezení, implementovali jsme algoritmus, který se inspiruje technikami vizualizace vektorových polí a používá shlukované trojrozměrné šipky ke kódování metrik. S důrazem na vizuální stránku jsme tento algoritmus implementovali v několika druzích vizualizací v rámci experimentální aplikace MeshDiff. Rovněž jsme provedli uživatelskou studii jak existujících, tak nově navržených vizualizací, abychom porovnali jejich kvalitu v různých situacích a prozkoumali, jak lze vizualizace v budoucnu vylepšit.

\end{document}