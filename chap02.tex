\chapter{Implementation}

In-depth explanation of the solution. (Developer documentation.)

\begin{itemize}
\item Thoroughly describe the clustering techniques used
\item Thoroughly describe the new visualization methods
\item Mention how it is all put together in the application
\item Talk about loading/saving parameters/visualizations
\end{itemize}

\begin{algorithm}[H]
\caption{Clustering}
\begin{algorithmic}[1]

\Require ClusterSet c
\Statex
\For{all cells cell\textsubscript{i} \textbf{in} dataset}
	\State c = makeCluster(cell\textsubscript{i});
    \State set level of c to 0;
    \State add c to s;
\EndFor
\Statex
\For{all clusters c\textsubscript{i} in s}
	\For{all clusters c\textsubscript{j} neighbours of c\textsubscript{i}}
    	\State e = clusteringError(c\textsubscript{i}, c\textsubscript{j});
        \State insert pair (c\textsubscript{i}, c\textsubscript{j}) in increasing order of error e in a hash table;
        \State mark c\textsubscript{i} and c\textsubscript{j} as NOT\_CLUSTERED;
    \EndFor
\EndFor
\Statex
\State int i = 0;
\For{all pairs (c\textsubscript{i}, c\textsubscript{j}) in increasing order of error in the hash table}
	\If{both c\textsubscript{i} and c\textsubscript{j} are NOT\_CLUSTERED}
    	\State c = mergeClusters(c\textsubscript{i}, c\textsubscript{j});
        set level of c to l++;
        mark c\textsubscript{i} and c\textsubscript{j} as CLUSTERED;
        \For{all neighbors n\textsubscript{i} of c}
        	\State e = clusteringError(c, n\textsubscript{i});
            insert pair (c, n\textsubscript{i}) in order in hash table
        \EndFor
    \EndIf
\EndFor
\Statex
\Return c as root of tree
\end{algorithmic}
\end{algorithm}