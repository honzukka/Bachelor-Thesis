\subsection{Question 1}
\label{attch:complete_study_results-question1}

\begin{center}{\it Which face has a larger nose?}\end{center}

\begin{figure}[h]
\centering
\begin{subfigure}{0.49\textwidth}
\includegraphics[width=\textwidth]{./img-study/pair2.PNG}
\caption{}
\label{fig:study-0-2}
\end{subfigure}
\begin{subfigure}{0.49\textwidth}
\includegraphics[width=\textwidth]{./img-study/pair4.PNG}
\caption{}
\label{fig:study-0-4}
\end{subfigure}

\begin{subfigure}{0.49\textwidth}
\includegraphics[width=\textwidth]{./img-study/pair1.PNG}
\caption{}
\label{fig:study-0-1}
\end{subfigure}
\begin{subfigure}{0.49\textwidth}
\includegraphics[width=\textwidth]{./img-study/pair3.PNG}
\caption{}
\label{fig:study-0-3}
\end{subfigure}
\caption{Visualizations shown for question 1}
\end{figure}
\medskip

{\bf Expected best visualization:} \ref{fig:study-0-4}
\medskip

{\bf Expected answer:} {\it N/A}
\medskip

{\bf Collected answers:}

\begin{center}
\begin{tabular}{| c | c | c | c | c |}
	\hline
	Visualization & \ref{fig:study-0-2} & \ref{fig:study-0-4} & \ref{fig:study-0-1} & \ref{fig:study-0-3}\\ \hline
	\(\widehat{t}(p, q_1, V_k)\) & 20.46 & 26.19 & 20.26 & 21.56\\ \hline
	\multicolumn{5}{|c|}{\bf Answers} \\ \hline
	Left & 7 & 8 & 6 & 4\\ \hline
	Right & 4 & 2 & 1 & 2\\ \hline
	NotSure & 2 & 1 & 0 & 0\\ \hline
	{\bf Total} & {\bf 13} & {\bf 11} & {\bf 7} & {\bf 6}\\ \hline
\end{tabular}
\end{center}

\begin{tabular}{l p{0.7\textwidth}}
	{\bf Commentary:} & The majority of participants thought the left to be larger. This shows their interpretation of the term ``larger'' (Fig. \ref{fig:study-0-1}) and that the visualizations favor this interpretation. We did not choose an expected answer because while the left nose is thicker, the right nose is longer. \\
\end{tabular}
 
\clearpage

\subsection{Question 2}
\label{attch:complete_study_results-question2}

\begin{center}{\it Does the chin stick out more to the front in the right face than in the left face?}\end{center}

\begin{figure}[h]
\centering
\begin{subfigure}{0.49\textwidth}
\includegraphics[width=\textwidth]{./img-study/pair3.PNG}
\caption{}
\label{fig:study-1-3}
\end{subfigure}
\begin{subfigure}{0.49\textwidth}
\includegraphics[width=\textwidth]{./img-study/pair1.PNG}
\caption{}
\label{fig:study-1-1}
\end{subfigure}

\begin{subfigure}{0.49\textwidth}
\includegraphics[width=\textwidth]{./img-study/pair2.PNG}
\caption{}
\label{fig:study-1-2}
\end{subfigure}
\begin{subfigure}{0.49\textwidth}
\includegraphics[width=\textwidth]{./img-study/pair4.PNG}
\caption{}
\label{fig:study-1-4}
\end{subfigure}
\caption{Visualizations shown for question 2}
\end{figure}
\medskip

{\bf Expected best visualization:} \ref{fig:study-1-4}
\medskip

{\bf Expected answer:} {\it No}
\medskip

{\bf Collected answers:}

\begin{center}
\begin{tabular}{| c | c | c | c | c |}
	\hline
	Visualization & \ref{fig:study-1-3} & \ref{fig:study-1-1} & \ref{fig:study-1-2} & \ref{fig:study-1-4}\\ \hline
	\(\widehat{t}(p, q_2, V_k)\) & 46.62 & 43.08 & 61.56 & 38.90\\ \hline
	\multicolumn{5}{|c|}{\bf Answers} \\ \hline
	Yes & 10 & 5 & 6 & 4\\ \hline
	\rowcolor{yellow!30} No & 3 & 4 & 1 & 2\\ \hline
	NotSure & 0 & 2 & 0 & 0\\ \hline
	{\bf Total} & {\bf 13} & {\bf 11} & {\bf 7} & {\bf 6}\\ \hline
\end{tabular}
\end{center}

\begin{tabular}{l p{0.7\textwidth}}
	{\bf Commentary:} & The participants have agreed that the answer should be ``Yes''. We assume that when only colors were shown, the difference seemed quite large in favor of the right face. When arrows were shown, we believe that some participants thought there was no difference, while others have seen a slight difference in favor of the right face. Our original view was that there was close to no difference related to the question. \\
\end{tabular}

\clearpage

\subsection{Question 3}
\label{attch:complete_study_results-question3}

\begin{center}{\it Where is the most significant difference between the two faces?}\end{center}

\begin{figure}[h]
\centering
\begin{subfigure}{0.49\textwidth}
\includegraphics[width=\textwidth]{./img-study/pair6.PNG}
\caption{}
\label{fig:study-2-6}
\end{subfigure}
\begin{subfigure}{0.49\textwidth}
\includegraphics[width=\textwidth]{./img-study/pair8.PNG}
\caption{}
\label{fig:study-2-8}
\end{subfigure}

\begin{subfigure}{0.49\textwidth}
\includegraphics[width=\textwidth]{./img-study/pair5.PNG}
\caption{}
\label{fig:study-2-5}
\end{subfigure}
\begin{subfigure}{0.49\textwidth}
\includegraphics[width=\textwidth]{./img-study/pair7.PNG}
\caption{}
\label{fig:study-2-7}
\end{subfigure}
\caption{Visualizations shown for question 3}
\end{figure}
\medskip

{\bf Expected best visualization:} \ref{fig:study-2-5}
\medskip

{\bf Expected answer:} {\it Right Cheek}
\medskip

{\bf Collected answers:}

\begin{center}
\begin{tabular}{| c | c | c | c | c |}
	\hline
	Visualization & \ref{fig:study-2-6} & \ref{fig:study-2-8} & \ref{fig:study-2-5} & \ref{fig:study-2-7}\\ \hline
	\(\widehat{t}(p, q_3, V_k)\) & 41.87 & 37.36 & 31.29 & 58.84\\ \hline
	\multicolumn{5}{|c|}{\bf Answers} \\ \hline
	LeftCheek & 2 & 1 & 3 & 1\\ \hline
	Forehead & 5 & 4 & 0 & 1\\ \hline
	\rowcolor{yellow!30} RightCheek & 2 & 4 & 2 & 1\\ \hline
	Nose & 1 & 0 & 0 & 0\\ \hline
	NotSure & 2 & 2 & 1 & 1\\ \hline
	Chin & 1 & 0 & 0 & 0\\ \hline
	Mouth & 0 & 0 & 1 & 2\\ \hline
	{\bf Total} & {\bf 13} & {\bf 11} & {\bf 7} & {\bf 6}\\ \hline
\end{tabular}
\end{center}

\begin{tabular}{l p{0.7\textwidth}}
	{\bf Commentary:} & Based on the answers provided for visualization \ref{fig:study-2-5}, we assume that the participants have confused ``left'' and ``right'' in this question. Visualization \ref{fig:study-2-8} did not exclude areas close to the edge of the mesh where a lot of error is accumulated by the way the mesh is cut. \\
\end{tabular}

\clearpage

\subsection{Question 4}
\label{attch:complete_study_results-question4}

\begin{center}{\it When examining the differences moving from the left face to the right face, what is their main direction in the area below the nose?}\end{center}

\begin{figure}[h]
\centering
\begin{subfigure}{0.49\textwidth}
\includegraphics[width=\textwidth]{./img-study/pair10.PNG}
\caption{}
\label{fig:study-3-10}
\end{subfigure}
\begin{subfigure}{0.49\textwidth}
\includegraphics[width=\textwidth]{./img-study/pair7.PNG}
\caption{}
\label{fig:study-3-7}
\end{subfigure}

\begin{subfigure}{0.49\textwidth}
\includegraphics[width=\textwidth]{./img-study/pair9.PNG}
\caption{}
\label{fig:study-3-9}
\end{subfigure}
\begin{subfigure}{0.49\textwidth}
\includegraphics[width=\textwidth]{./img-study/pair6.PNG}
\caption{}
\label{fig:study-3-6}
\end{subfigure}
\caption{Visualizations shown for question 4}
\end{figure}
\medskip

{\bf Expected best visualization:} \ref{fig:study-3-9}
\medskip

{\bf Expected answer:} {\it Down}
\medskip

{\bf Collected answers:}

\begin{center}
\begin{tabular}{| c | c | c | c | c |}
	\hline
	Visualization & \ref{fig:study-3-10} & \ref{fig:study-3-7} & \ref{fig:study-3-9} & \ref{fig:study-3-6}\\ \hline
	\(\widehat{t}(p, q_4, V_k)\) & 48.19 & 53.41 & 47.80 & 55.13\\ \hline
	\multicolumn{5}{|c|}{\bf Answers} \\ \hline
	\rowcolor{yellow!30} Down & 5 & 2 & 2 & 1\\ \hline
	Right & 1 & 0 & 0 & 1\\ \hline
	In & 2 & 2 & 1 & 0\\ \hline
	NotSure & 2 & 1 & 2 & 2\\ \hline
	Left & 1 & 1 & 0 & 0\\ \hline
	Out & 2 & 5 & 0 & 1\\ \hline
	Up & 0 & 0 & 2 & 1\\ \hline
	{\bf Total} & {\bf 13} & {\bf 11} & {\bf 7} & {\bf 6}\\ \hline
\end{tabular}
\end{center}
\clearpage

\subsection{Question 5}
\label{attch:complete_study_results-question5}

\begin{center}{\it Does the left cheek stick out more to the front in the right face than in the left face?}\end{center}

\begin{figure}[h]
\centering
\begin{subfigure}{0.49\textwidth}
\includegraphics[width=\textwidth]{./img-study/pair12.PNG}
\caption{}
\label{fig:study-4-12}
\end{subfigure}
\begin{subfigure}{0.49\textwidth}
\includegraphics[width=\textwidth]{./img-study/pair14.PNG}
\caption{}
\label{fig:study-4-14}
\end{subfigure}

\begin{subfigure}{0.49\textwidth}
\includegraphics[width=\textwidth]{./img-study/pair11.PNG}
\caption{}
\label{fig:study-4-11}
\end{subfigure}
\begin{subfigure}{0.49\textwidth}
\includegraphics[width=\textwidth]{./img-study/pair13.PNG}
\caption{}
\label{fig:study-4-13}
\end{subfigure}
\caption{Visualizations shown for question 5}
\end{figure}
\medskip

{\bf Expected best visualization:} \ref{fig:study-4-11}
\medskip

{\bf Expected answer:} {\it No}
\medskip

{\bf Collected answers:}

\begin{center}
\begin{tabular}{| c | c | c | c | c |}
	\hline
	Visualization & \ref{fig:study-4-12} & \ref{fig:study-4-14} & \ref{fig:study-4-11} & \ref{fig:study-4-13}\\ \hline
	\(\widehat{t}(p, q_5, V_k)\) & 40.74 & 43.85 & 49.87 & 31.33\\ \hline
	\multicolumn{5}{|c|}{\bf Answers} \\ \hline
	Yes & 8 & 5 & 6 & 3\\ \hline
	\rowcolor{yellow!30} No & 5 & 6 & 1 & 3\\ \hline
	{\bf Total} & {\bf 13} & {\bf 11} & {\bf 7} & {\bf 6}\\ \hline
\end{tabular}
\end{center}

\begin{tabular}{l p{0.7\textwidth}}
	{\bf Commentary:} & We have based our expected answer on the thin green stripe on the left face in color visualizations. The opposing answers might have stemmed from the fact that ``left'' and ``right'' were confused again or that a larger area was understood as a cheek and the difference was interpreted more globally. \\
\end{tabular}

\clearpage

\subsection{Question 6}
\label{attch:complete_study_results-question6}

\begin{center}{\it Which face has a longer nose?}\end{center}

\begin{figure}[h]
\centering
\begin{subfigure}{0.49\textwidth}
\includegraphics[width=\textwidth]{./img-study/pair13.PNG}
\caption{}
\label{fig:study-5-13}
\end{subfigure}
\begin{subfigure}{0.49\textwidth}
\includegraphics[width=\textwidth]{./img-study/pair11.PNG}
\caption{}
\label{fig:study-5-11}
\end{subfigure}

\begin{subfigure}{0.49\textwidth}
\includegraphics[width=\textwidth]{./img-study/pair12.PNG}
\caption{}
\label{fig:study-5-12}
\end{subfigure}
\begin{subfigure}{0.49\textwidth}
\includegraphics[width=\textwidth]{./img-study/pair14.PNG}
\caption{}
\label{fig:study-5-14}
\end{subfigure}
\caption{Visualizations shown for question 6}
\end{figure}
\medskip

{\bf Expected best visualization:} \ref{fig:study-5-11}
\medskip

{\bf Expected answer:} {\it Left}
\medskip

{\bf Collected answers:}

\begin{center}
\begin{tabular}{| c | c | c | c | c |}
	\hline
	Visualization & \ref{fig:study-5-13} & \ref{fig:study-5-11} & \ref{fig:study-5-12} & \ref{fig:study-5-14}\\ \hline
	\(\widehat{t}(p, q_6, V_k)\) & 33.37 & 32.34 & 23.85 & 29.71\\ \hline
	\multicolumn{5}{|c|}{\bf Answers} \\ \hline
	NotSure & 2 & 1 & 0 & 0\\ \hline
	Right & 3 & 5 & 3 & 1\\ \hline
	\rowcolor{yellow!30} Left & 8 & 5 & 4 & 5\\ \hline
	{\bf Total} & {\bf 13} & {\bf 11} & {\bf 7} & {\bf 6}\\ \hline
\end{tabular}
\end{center}
\clearpage

\subsection{Question 7}
\label{attch:complete_study_results-question7}

\begin{center}{\it Which face has larger cheekbones?}\end{center}

\begin{figure}[h]
\centering
\begin{subfigure}{0.49\textwidth}
\includegraphics[width=\textwidth]{./img-study/pair17.PNG}
\caption{}
\label{fig:study-6-17}
\end{subfigure}
\begin{subfigure}{0.49\textwidth}
\includegraphics[width=\textwidth]{./img-study/pair15.PNG}
\caption{}
\label{fig:study-6-15}
\end{subfigure}

\begin{subfigure}{0.49\textwidth}
\includegraphics[width=\textwidth]{./img-study/pair18.PNG}
\caption{}
\label{fig:study-6-18}
\end{subfigure}
\begin{subfigure}{0.49\textwidth}
\includegraphics[width=\textwidth]{./img-study/pair16.PNG}
\caption{}
\label{fig:study-6-16}
\end{subfigure}
\caption{Visualizations shown for question 7}
\end{figure}
\medskip

{\bf Expected best visualization:} \ref{fig:study-6-17}
\medskip

{\bf Expected answer:} {\it Right}
\medskip

{\bf Collected answers:}

\begin{center}
\begin{tabular}{| c | c | c | c | c |}
	\hline
	Visualization & \ref{fig:study-6-17} & \ref{fig:study-6-15} & \ref{fig:study-6-18} & \ref{fig:study-6-16}\\ \hline
	\(\widehat{t}(p, q_7, V_k)\) & 16.11 & 24.59 & 18.47 & 15.01\\ \hline
	\multicolumn{5}{|c|}{\bf Answers} \\ \hline
	\rowcolor{yellow!30} Right & 11 & 4 & 6 & 5\\ \hline
	Left & 2 & 4 & 0 & 1\\ \hline
	NotSure & 0 & 3 & 1 & 0\\ \hline
	{\bf Total} & {\bf 13} & {\bf 11} & {\bf 7} & {\bf 6}\\ \hline
\end{tabular}
\end{center}
\clearpage

\subsection{Question 8}
\label{attch:complete_study_results-question8}

\begin{center}{\it Which face has a larger eyebrow bone?}\end{center}

\begin{figure}[h]
\centering
\begin{subfigure}{0.49\textwidth}
\includegraphics[width=\textwidth]{./img-study/pair20.PNG}
\caption{}
\label{fig:study-7-20}
\end{subfigure}
\begin{subfigure}{0.49\textwidth}
\includegraphics[width=\textwidth]{./img-study/pair22.PNG}
\caption{}
\label{fig:study-7-22}
\end{subfigure}

\begin{subfigure}{0.49\textwidth}
\includegraphics[width=\textwidth]{./img-study/pair19.PNG}
\caption{}
\label{fig:study-7-19}
\end{subfigure}
\begin{subfigure}{0.49\textwidth}
\includegraphics[width=\textwidth]{./img-study/pair21.PNG}
\caption{}
\label{fig:study-7-21}
\end{subfigure}
\caption{Visualizations shown for question 8}
\end{figure}
\medskip

{\bf Expected best visualization:} \ref{fig:study-7-21}
\medskip

{\bf Expected answer:} {\it Right}
\medskip

{\bf Collected answers:}

\begin{center}
\begin{tabular}{| c | c | c | c | c |}
	\hline
	Visualization & \ref{fig:study-7-20} & \ref{fig:study-7-22} & \ref{fig:study-7-19} & \ref{fig:study-7-21}\\ \hline
	\(\widehat{t}(p, q_8, V_k)\) & 17.00 & 16.25 & 24.09 & 16.03\\ \hline
	\multicolumn{5}{|c|}{\bf Answers} \\ \hline
	\rowcolor{yellow!30} Right & 12 & 8 & 4 & 5\\ \hline
	NotSure & 1 & 0 & 0 & 0\\ \hline
	Left & 0 & 3 & 3 & 1\\ \hline
	{\bf Total} & {\bf 13} & {\bf 11} & {\bf 7} & {\bf 6}\\ \hline
\end{tabular}
\end{center}
\clearpage

\subsection{Question 9}
\label{attch:complete_study_results-question9}

\begin{center}{\it Where are the two faces the most similar?}\end{center}

\begin{figure}[h]
\centering
\begin{subfigure}{0.49\textwidth}
\includegraphics[width=\textwidth]{./img-study/pair19.PNG}
\caption{}
\label{fig:study-8-19}
\end{subfigure}
\begin{subfigure}{0.49\textwidth}
\includegraphics[width=\textwidth]{./img-study/pair21.PNG}
\caption{}
\label{fig:study-8-21}
\end{subfigure}

\begin{subfigure}{0.49\textwidth}
\includegraphics[width=\textwidth]{./img-study/pair23.PNG}
\caption{}
\label{fig:study-8-23}
\end{subfigure}
\begin{subfigure}{0.49\textwidth}
\includegraphics[width=\textwidth]{./img-study/pair20.PNG}
\caption{}
\label{fig:study-8-20}
\end{subfigure}
\caption{Visualizations shown for question 9}
\end{figure}
\medskip

{\bf Expected best visualization:} \ref{fig:study-8-23}
\medskip

{\bf Expected answer:} {\it Nose}
\medskip

{\bf Collected answers:}

\begin{center}
\begin{tabular}{| c | c | c | c | c |}
	\hline
	Visualization & \ref{fig:study-8-19} & \ref{fig:study-8-21} & \ref{fig:study-8-23} & \ref{fig:study-8-20}\\ \hline
	\(\widehat{t}(p, q_9, V_k)\) & 47.23 & 51.02 & 36.77 & 42.21\\ \hline
	\multicolumn{5}{|c|}{\bf Answers} \\ \hline
	NotSure & 2 & 1 & 1 & 2\\ \hline
	LeftCheek & 2 & 0 & 0 & 0\\ \hline
	Chin & 6 & 0 & 1 & 0\\ \hline
	RightCheek & 3 & 2 & 1 & 2\\ \hline
	Mouth & 0 & 2 & 1 & 0\\ \hline
	\rowcolor{yellow!30} Nose & 0 & 5 & 3 & 1\\ \hline
	Forehead & 0 & 1 & 0 & 1\\ \hline
	{\bf Total} & {\bf 13} & {\bf 11} & {\bf 7} & {\bf 6}\\ \hline
\end{tabular}
\end{center}
\clearpage

\subsection{Question 10}
\label{attch:complete_study_results-question10}

\begin{center}{\it When examining the differences moving from the left face to the right face, what is their main direction overall?}\end{center}

\begin{figure}[h]
\centering
\begin{subfigure}{0.49\textwidth}
\includegraphics[width=\textwidth]{./img-study/pair21.PNG}
\caption{}
\label{fig:study-9-21}
\end{subfigure}
\begin{subfigure}{0.49\textwidth}
\includegraphics[width=\textwidth]{./img-study/pair19.PNG}
\caption{}
\label{fig:study-9-19}
\end{subfigure}

\begin{subfigure}{0.49\textwidth}
\includegraphics[width=\textwidth]{./img-study/pair20.PNG}
\caption{}
\label{fig:study-9-20}
\end{subfigure}
\begin{subfigure}{0.49\textwidth}
\includegraphics[width=\textwidth]{./img-study/pair24.PNG}
\caption{}
\label{fig:study-9-24}
\end{subfigure}
\caption{Visualizations shown for question 10}
\end{figure}
\medskip

{\bf Expected best visualization:} \ref{fig:study-9-24}
\medskip

{\bf Expected answer:} {\it Down}
\medskip

{\bf Collected answers:}

\begin{center}
\begin{tabular}{| c | c | c | c | c |}
	\hline
	Visualization & \ref{fig:study-9-21} & \ref{fig:study-9-19} & \ref{fig:study-9-20} & \ref{fig:study-9-24}\\ \hline
	\(\widehat{t}(p, q_10, V_k)\) & 50.42 & 33.90 & 48.06 & 53.28\\ \hline
	\multicolumn{5}{|c|}{\bf Answers} \\ \hline
	NotSure & 4 & 2 & 2 & 1\\ \hline
	\rowcolor{yellow!30} Down & 3 & 1 & 1 & 2\\ \hline
	In & 2 & 1 & 1 & 0\\ \hline
	Out & 3 & 4 & 2 & 0\\ \hline
	Up & 1 & 2 & 0 & 2\\ \hline
	Left & 0 & 1 & 0 & 0\\ \hline
	Right & 0 & 0 & 1 & 1\\ \hline
	{\bf Total} & {\bf 13} & {\bf 11} & {\bf 7} & {\bf 6}\\ \hline
\end{tabular}
\end{center}
\clearpage

