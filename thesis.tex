%%% The main file. It contains definitions of basic parameters and includes all other parts.

%% Settings for single-side (simplex) printing
% Margins: left 40mm, right 25mm, top and bottom 25mm
% (but beware, LaTeX adds 1in implicitly)
\documentclass[12pt,a4paper]{report}
\setlength\textwidth{145mm}
\setlength\textheight{247mm}
\setlength\oddsidemargin{15mm}
\setlength\evensidemargin{15mm}
\setlength\topmargin{0mm}
\setlength\headsep{0mm}
\setlength\headheight{0mm}
% \openright makes the following text appear on a right-hand page
\let\openright=\clearpage

%% Settings for two-sided (duplex) printing
% \documentclass[12pt,a4paper,twoside,openright]{report}
% \setlength\textwidth{145mm}
% \setlength\textheight{247mm}
% \setlength\oddsidemargin{14.2mm}
% \setlength\evensidemargin{0mm}
% \setlength\topmargin{0mm}
% \setlength\headsep{0mm}
% \setlength\headheight{0mm}
% \let\openright=\cleardoublepage

%% Fix strange behavior of \textsubscript in fancyvrb mode
\let\tmpa\textsubscript
\DeclareTextCommandDefault{\textsubscript}{\tmpa}

%% Generate PDF/A-2u
\usepackage[a-2u]{pdfx}

%% Character encoding: usually latin2, cp1250 or utf8:
\usepackage[utf8]{inputenc}

%% Prefer Latin Modern fonts
\usepackage{lmodern}

%% Further useful packages (included in most LaTeX distributions)
\usepackage{amsmath}        % extensions for typesetting of math
\usepackage{amsfonts}       % math fonts
\usepackage{amsthm}         % theorems, definitions, etc.
\usepackage{bbding}         % various symbols (squares, asterisks, scissors, ...)
\usepackage{bm}             % boldface symbols (\bm)
\usepackage{graphicx}       % embedding of pictures
\usepackage{fancyvrb}       % improved verbatim environment
\usepackage{natbib}         % citation style AUTHOR (YEAR), or AUTHOR [NUMBER]
\usepackage[nottoc]{tocbibind} % makes sure that bibliography and the lists
			    % of figures/tables are included in the table
			    % of contents
\usepackage{dcolumn}        % improved alignment of table columns
\usepackage{booktabs}       % improved horizontal lines in tables
\usepackage{paralist}       % improved enumerate and itemize
\usepackage[usenames]{xcolor}  % typesetting in color

%% My packages
\usepackage{verbatim}					% multi-line comments
\usepackage[list=true]{subcaption}		% subfigures
\usepackage[noend]{algpseudocode}		% part of the algorithmicx bundle for pseudocode
\usepackage{algorithm}					% part of the algorithmicx bundle for pseudocode
\usepackage{float}						% for floating classes anchor (like algorithm)

%% Set the font of algorithms
\makeatletter
\algrenewcommand\ALG@beginalgorithmic{\ttfamily}
\makeatother

%% Allow verbatim text in footnotes
\VerbatimFootnotes

%%% Basic information on the thesis

% Thesis title in English (exactly as in the formal assignment)
\def\ThesisTitle{Visualization of the difference between two triangle meshes}

% Author of the thesis
\def\ThesisAuthor{Jan Horešovský}

% Year when the thesis is submitted
\def\YearSubmitted{2018}

% Name of the department or institute, where the work was officially assigned
% (according to the Organizational Structure of MFF UK in English,
% or a full name of a department outside MFF)
\def\Department{Department of Software and Computer Science Education}

% Is it a department (katedra), or an institute (ústav)?
\def\DeptType{Department}

% Thesis supervisor: name, surname and titles
\def\Supervisor{RNDr. Josef Pelikán}

% Supervisor's department (again according to Organizational structure of MFF)
\def\SupervisorsDepartment{Department of Software and Computer Science Education}

% Study programme and specialization
\def\StudyProgramme{Computer Science}
\def\StudyBranch{Programming and Software Systems}

% An optional dedication: you can thank whomever you wish (your supervisor,
% consultant, a person who lent the software, etc.)
\def\Dedication{%
I would like to thank Ida, Lenka and Petr for their support and my supervisor RNDr. Josef Pelikán and my consultant Mgr. Ján Dupej for their feedback and ideas. A special thanks belongs to all volunteers who have participated in the user study.
}

% Abstract (recommended length around 80-200 words; this is not a copy of your thesis assignment!)
\def\Abstract{%
Morphome3cs, a program created by researchers at Charles University, allows users to process and analyze 3D data, mostly of anthropological and archaeological origin. Morphome3cs is also able to visualize the difference between two triangle meshes by encoding various metrics into vertex color. However, this one-dimensional information is not enough to display multiple metrics at the same time. To overcome this limitation, we implemented an algorithm which employs the techniques of vector field visualization and uses clustered 3D arrows to encode the metrics. Focusing on visual appearance, we applied it in several types of visualizations in an experimental application called MeshDiff. In order to find out if these new approaches should be incorporated into Morphome3cs, we also carried out a user study which evaluates the quality of the new visualizations in various use cases.
}

% 3 to 5 keywords (recommended), each enclosed in curly braces
\def\Keywords{%
{visualization} {triangle-mesh} {difference} {clustering} {vector-field}
}

%% The hyperref package for clickable links in PDF and also for storing
%% metadata to PDF (including the table of contents).
%% Most settings are pre-set by the pdfx package.
\hypersetup{unicode}
\hypersetup{breaklinks=true}

% Definitions of macros (see description inside)
\include{macros}

% Title page and various mandatory informational pages
\begin{document}
\include{title}

%%% A page with automatically generated table of contents of the bachelor thesis

\tableofcontents

%%% Each chapter is kept in a separate file
\chapter*{Introduction}
\addcontentsline{toc}{chapter}{Introduction}

Introduce the problem, explain why it is interesting and outline the goal of the thesis.

\begin{itemize}
\item Introduce the problem
\item Describe what has already been done
\item {\bf Emphasize what we want to achieve}
\item Mention the structure of the thesis
\end{itemize}

More in detail:

\begin{itemize}
\item The problem is that current visualization methods in Morphome3cs which work with homologous meshes are not sufficient.
\item Various metrics are being visualized by vertex color. Also mention related difference visualizations and that they all solve a slightly different problem (no homologous meshes) and mostly also just use color.
\item {\bf The goal is to come up with a good arrow-based visualization and experiment with the visual appearance to get a satisfactory result.}
\item (Thesis structure)
\end{itemize}

\section*{Morphome3cs}

\section*{Applications of Mesh Difference Visualization}
%%-----------------------------------------------------------------------------------------
%%-----------------------------------------------------------------------------------------
\section*{Mesh Difference Visualization in Morphome3cs and Elsewhere}

Morphome3cs is able to generate a homologous\footnotemark pair of triangle meshes from two arbitrary meshes and uses it to compute and visualize the difference. Currently, Morphome3cs is able to produce color-based visualizations of multiple difference metrics. These metrics are:

\begin{itemize}
\item Vertex distance (fig. \ref{fig:morpho_example})
\item Vertex distance projected into the surface normal
\item Angle between corresponding surface normals
\item FESA\footnotemark
\item Curvature difference
\end{itemize}

The disadvantage of these color-based visualizations is that they fail to capture multi-dimensional information. For example, when using vertex distance as a metric, it is impossible to encode both magnitude and direction into color at the same time while maintaining visual clarity.

\begin{figure}[h]
\centering
\includegraphics[width=0.5\textwidth]{./img/morpho-example01.PNG}
\caption{Morphome3cs - Vertex difference visualization}
\label{fig:morpho_example}
\end{figure}

Other approaches can be found for example in MeshLab (fig. \ref{fig:meshlab_example}) and CloudCompare (fig. \ref{fig:cloudcompare_example}). Both programs, however, also use only color-based visualizations. Both programs work with arbitrary triangle meshes and CloudCompare can also work with point clouds, register them and visualize the difference between them.

\begin{figure}[h]
\centering
	\begin{subfigure}{0.3\textwidth}
	\includegraphics[width=\textwidth]{./img/meshlab-example01.PNG}
    \caption{MeshLab - Hausdorff Distance visualization}
    \label{fig:meshlab_example}
	\end{subfigure}
    \qquad
    \begin{subfigure}{0.3\textwidth}
	\includegraphics[width=\textwidth]{./img/cloudcompare-example01.PNG}
    \caption{CloudCompare - Vertex distance visualization}
    \label{fig:cloudcompare_example}
	\end{subfigure}
\caption{Visualizations in MeshLab and CloudCompare}
\end{figure}

\addtocounter{footnote}{-2}
\stepcounter{footnote}\footnotetext{Two triangle meshes are homologous if they have the same number of vertices and there is a one-to-one mapping between them. Vertices are numbered and vertex \(v_i \in Mesh_1\) corresponds to vertex \(v_i \in Mesh_2\).}
\stepcounter{footnote}\footnotetext{{\it Finite Element Surface Analysis}, captures the difference between corresponding triangle areas}
%%-----------------------------------------------------------------------------------------
%%-----------------------------------------------------------------------------------------
\section*{Our Goal}

\section*{Thesis Structure}

\chapter{Problem Analysis \& Solution}

Present the solution.

%%-----------------------------------------------------------------------------------------
%%-----------------------------------------------------------------------------------------
\section{Seeing Mesh Difference as a Vector Field}

In our new visualizations, arrows will be used to represent the difference between two triangle meshes. Each arrow will be internally represented by a position and a direction vector. Together, they will form a discrete and bounded vector field. This is a very important abstraction because vector field visualization is a very rich area which finds applications in engineering, molecular modeling and computational fluid dynamics. Therefore, there exist many scientific papers studying this topic, such as \citet{Telea99}, \citet{Garcke00}, \citet{Du04} or \citet{Peng12}.

When visualizing a vector field, it is necessary to use clustering on the vectors to obtain a simplified representation, otherwise the visualization becomes too cluttered (see fig. \ref{fig:meshdiff_unclustered}). Clustering therefore determines to a large extent what the final visualization will look like.

\begin{figure}[h]
\centering
\includegraphics[width=\textwidth]{./img/meshdiff-unclustered_arrows.PNG}
\caption{MeshDiff - Vertex distance visualized by unclustered arrows}
\label{fig:meshdiff_unclustered}
\end{figure}
%%-----------------------------------------------------------------------------------------
%% SECTION
%%-----------------------------------------------------------------------------------------
\section{Vector Field Clustering}

As clustering is a very subjective task (see fig. \ref{fig:clustering_subjectivity}), each of the above mentioned papers has their own way of dealing with it.

\begin{figure}[h]
\centering
\includegraphics[width=0.6\textwidth]{./img/clustering_subjectivity.png}
\caption{The subjectivity of clustering - are there two or four clusters in the image?}
\label{fig:clustering_subjectivity}
\end{figure}

%%-----------------------------------------------------------------------------------------
\subsection{Overview of Vector Field Clustering Methods}

\citet{Telea99} use hierarchical (bottom-up) clustering where neighboring clusters with the smallest clustering error are merged. Each cluster has a representative vector and when merging, a weighted average of the two vectors is computed and assigned to the newly formed cluster. The paper introduced elliptic iso-error contours to compute the clustering error. The method is primarily aimed at 2D rectilinear vector fields but can be also used in 3D.

\citet{Garcke00} use a continuous clustering method based on the physical model of \citet{CahnHilliard58} which is used to describe phase separation and coarsening in binary alloys. This model is applied to vector field data which results in a diffusion problem rather than a splitting and merging problem. Such an algorithm also presumes an either 2D or 3D rectilinear grid.

\citet{Du04} use iterative (top-down) clustering where Voronoi regions are created around the initial cluster centers and a distance function is applied to each of them. The set of cluster centers which has the lowest value of the distance function is selected as the final cluster centers set. This method works with 2D and 3D rectilinear vector fields.

\citet{Peng12} use hierarchical clustering similar to \citet{Telea99}, only they use the GPU to compute the clustering by encoding a certain static view of a mesh into a rasterized image. The computation is then done for this specific image. To obtain the clustering error, a very simple formula is employed:

\begin{equation} \label{eq:clustering_error}
\bm{e}(C_1,C_2) = k_d \cdot \frac{d_{C_1C_2}}{d_{max}} + k_v \cdot \frac{v_{C_1C_2}}{v_{max}} + k_\alpha \cdot \frac{\alpha_{C_1C_2}}{\alpha_{max}} + k_m \cdot \frac{m_{C_1C_2}}{m_{max}}
\end{equation}

where \(k_d + k_v + k_\alpha + k_m = 1\). The other components are the following:

\begin{itemize}
\item \(d_{C_1C_2}\) is the Euclidean distance between the positions of the representative vectors of the clusters. The maximum distance \(d_{max}\) is the length of a diagonal of the geometry's bounding box
\item \(v_{C_1C_2}\) is the difference between the lengths of the representative vectors. The maximum velocity \(v_{max}\) is the largest length in the whole data set.
\item \(\alpha_{C_1C_2}\) is angle between the representative vectors. The maximum angle \(\alpha_{max} = 180^\circ\)
\item \(m_{C_1C_2}\) is the sum of the mesh resolutions of the two clusters. \(m_{max}\) is the largest value of \(m\) in the whole data set.
\end{itemize}

The mesh resolution component also differentiates this approach from all others because it represents an approximation of the density of the mesh in a given local area. Including it in the error formula assigns higher error to dense clusters which results in a larger amount of clusters (higher precision) in dense areas of the mesh and a smaller amount of clusters (lower precision) in sparse areas of the mesh. This method is therefore aimed at non-rectilinear 3D meshes.
%%-----------------------------------------------------------------------------------------
\subsection{Our Clustering Method}

For our clustering purposes, it is convenient to choose one of the presented methods and modify it if necessary. This approach saves time and gives us a baseline which we can improve upon. When selecting a clustering method, it is necessary to find a balance between simplicity and the ease of use (parameter tuning). We will lean towards simplicity in order to quickly obtain our baseline and only later look for improvements once we have a better idea of the actual performance of the selected method in our conditions.

While the hierarchical clustering of \citet{Telea99} is very simple and seems like a good candidate, its error function does not take into account the varying density of our triangle meshes and is also more demanding to use in 3D. In order to calculate the error function, one needs to find the elliptic iso-error contour of a given vector. This leads to cube-root equations which would have to be computed for every pair of clustering candidates. The error function of \citet{Peng12} (see eq. \ref{eq:clustering_error}), on the other hand, is both sensitive to non-uniform meshes and computationally simpler and more scalable into higher dimensions.

Lastly, we will not use the GPU-based clustering computation from \citet{Peng12} because we need to be able to view the resulting visualizations easily from various angles in real time. The basic approach in \citet{Telea99} computes the clustering once for the whole vector field.

Our clustering method will use the hierarchical algorithm from \citet{Telea99} and the error function \ref{eq:clustering_error} from \citet{Peng12}. Merging of two clusters will be done by computing a weighted average of their representative vectors.
%%-----------------------------------------------------------------------------------------
\subsection{Algorithm}

Here is the pseudocode of the hierarchical clustering algorithm as presented in \citet{Telea99}:

\begin{algorithm}[H]
\caption{Clustering}
\begin{algorithmic}[1]

\Require ClusterSet s
\Statex
\For{all cells cell\textsubscript{i} \textbf{in} dataset}
	\State c = makeCluster(cell\textsubscript{i});
    \State set level of c to 0;
    \State add c to s;
\EndFor
\Statex
\For{all clusters c\textsubscript{i} in s}
	\For{all clusters c\textsubscript{j} neighbours of c\textsubscript{i}}
    	\State e = clusteringError(c\textsubscript{i}, c\textsubscript{j});
        \State insert pair (c\textsubscript{i}, c\textsubscript{j}) in increasing order of error e in a hash table;
        \State mark c\textsubscript{i} and c\textsubscript{j} as NOT\_CLUSTERED;
    \EndFor
\EndFor
\Statex
\State int i = 0;
\For{all pairs (c\textsubscript{i}, c\textsubscript{j}) in increasing order of error in the hash table}
	\If{both c\textsubscript{i} and c\textsubscript{j} are NOT\_CLUSTERED}
    	\State c = mergeClusters(c\textsubscript{i}, c\textsubscript{j});
        set level of c to l++;
        mark c\textsubscript{i} and c\textsubscript{j} as CLUSTERED;
        \For{all neighbors n\textsubscript{i} of c}
        	\State e = clusteringError(c, n\textsubscript{i});
            insert pair (c, n\textsubscript{i}) in order in hash table
        \EndFor
    \EndIf
\EndFor
\Statex
\Return c as root of tree
\end{algorithmic}
\end{algorithm}

There are two important functions in the algorithm, \texttt{clusteringError()} and \texttt{mergeClusters()}. \texttt{clusteringError()} uses formula \ref{eq:clustering_error} to compute the clustering error and therefore directly influence which clustering candidates will be merged first. \texttt{mergeClusters()} constructs the representative vector of the new cluster by computing a weighted average of the representative vectors of the merged clusters where the weight is the geometrical area of the clusters.
%%-----------------------------------------------------------------------------------------
\subsection{Improvements}

We also made one modification of the clustering algorithm which can be optionally turned on in the experimental application MeshDiff.

When we consider the surface of a triangle mesh with our vector field placed in the vertices of the mesh, there are two types of vectors - those pointing "inside" the mesh and those pointing "out" of the mesh. We found out that it might be useful to add one more condition to the clustering process and only allow those clusters to be merged whose representative arrows point in the same direction as outlined above. We say that such vectors have the same orientation.

The result of such a clustering is then a forest instead of a tree. This may result in certain clusterings not covering the whole mesh which does not necessarily lower the quality of the resulting visualization but should nevertheless be taken into account.
%%-----------------------------------------------------------------------------------------
%% SECTION
%%-----------------------------------------------------------------------------------------
\section{Proposed Visualizations}

Since our new visualizations are mostly arrow-based, we have chosen to visualize the following distance metrics:

\begin{itemize}
\item Vertex distance
\item Vertex distance projected into surface normal
\end{itemize}

Following are the descriptions of the new visualizations.

%%-----------------------------------------------------------------------------------------
\subsection{Arrows}

Once a clustering of the mesh difference vector field is obtained, representative vectors of all clusters are visualized using 3D arrows. For this purpose we have prepared a simple 3D model of an arrow which is copied to the scene at a specific position and a specific scale given by the representative vector and the cluster it belongs to.

The length of the representative vector encodes the difference metric now averaged across the whole cluster. This length also influences the scale of the 3D arrow. Because the values of the metric can be very close to zero and because their value range is not generally very large, we have decided set a certain minimum scale and maximum scale, which is adjustable by the user, and map the metric values (vector lengths) to this interval. In general, such an approach gives a more visually pleasing and clear results, especially when the interval is chosen to be large enough.

The scale of the 3D arrows also reflect the geometrical area of the clusters. The larger the clusters are, the thicker the arrows. Areas are again mapped to a user-defined interval for clearer result. Large clusters are usually important because they represent a trend in a given area and users should be able to see them more easily and also distinguish them from less important arrows.

Lastly, and most importantly, the direction of the representative vector is directly reflected in the direction of the 3D arrow. We have therefore managed to encode three-dimensional information into our visualization.

\subsubsection{Expected Performance}

Arrow visualizations combined with clustering are expected to perform well in answering questions about general trends in large parts of a mesh. They are also expected to perform considerably better than color-based visualizations when asking about the direction of the difference, which is particularly important in cases when the difference forms a very small angle with the surface of the mesh.
%%-----------------------------------------------------------------------------------------
\subsection{Cluster Color}

There are situations when the user requires clustering in order to see how larger parts of the meshes differ from one another but the richness of information offered by the arrow visualization is redundant. Also, one might simply want to see how the clustering itself behaves. For these use case we introduce two color visualizations of clusters - random and metric-based.

Both of these visualizations need to be aware of which mesh vertices belong to a given cluster. Then either a random color is assigned to all vertices in a given cluster or a color based on the length of the representative vector of the cluster is assigned. The former case is basically the original color visualization of a metric, only applied on clusters.

\subsubsection{Expected Performance}

Random cluster color is expected to be used for the purpose of configuring the clustering parameters as the size and the location of the clusters is clearly visible in this case.

Metric-based cluster color is expected to perform well in cases when we have found the most important differences using a more sophisticated visualization and want to present those using a visualization which is as clear as possible.
%%-----------------------------------------------------------------------------------------
\subsection{Thresholding}

For all types of metric-based visualizations, including the original color-based ones, we present thresholding. Each vertex or cluster which does not have a sufficiently high metric value assigned to it is excluded from the visualization. This method therefore works as a high-pass filter on the visualization.

\subsubsection{Expected Performance}

Thresholding is expected to enhance the effect of metric-based cluster color in that it is expected to perform very well when segmenting and emphasizing a previously discovered difference which is of importance to the user. It is also expected to help answer questions about the largest differences in the mesh.
%%-----------------------------------------------------------------------------------------
\subsection{Combined Visualizations}

The last visualization type presented in this thesis is the combination of color and arrow visualizations.

\subsubsection{Expected Performance}

Combined visualizations complement each other in areas when only one visualization is not sufficiently clear. Color visualizations are expected to highlight the dimension of the metric we are interested in the most, for example vertex distance magnitude, while arrow visualization are expected to carry the other dimensions like direction and cluster size. This method is expected to bring a balance between clarity and information richness.
%%-----------------------------------------------------------------------------------------
%% SECTION
%%-----------------------------------------------------------------------------------------
\section{Effect of Clustering Parameters on the Visualizations}

(include visualizer parameters in the descriptions above)

The function used for computing the clustering error (see eq. \ref{eq:clustering_error}) has four parameters:

\begin{itemize}
\item Direction weight
\item Position weight
\item Magnitude weight
\item Resolution weight
\end{itemize}

A specific configuration of these parameters can influence the outcome of the clustering process considerably. Setting the value of one of them higher than the others will cause the proportion of the corresponding part of the clustering error be higher than the others. This will make that part of the error more significant and cause cluster pair with low values in this area merge much easier than all the other cluster pairs.

%%-----------------------------------------------------------------------------------------
\subsection{Direction Weight}

High direction weight causes clusters whose representative vectors have similar directions merge easily regardless of their difference in position, magnitude and resolution. This results in uneven cluster sizes. It forms many small clusters in areas of high surface curvature and large clusters in flat areas. Therefore, it mostly captures the high-curvature changes of shape.
%%-----------------------------------------------------------------------------------------
\subsection{Position Weight}

Setting the position weight higher than others results in clusters of even size where each of them represents the overall difference in a certain area regardless of the variety of directions and magnitudes in that area.
%%-----------------------------------------------------------------------------------------
\subsection{Magnitude Weight}

Magnitude weight can play a significant role when using the metric-based cluster color visualization because its high value will highlight iso-magnitude contours. Such an approach can be useful when grouping and segmenting areas with a certain absolute value of the difference metric.
%%-----------------------------------------------------------------------------------------
\subsection{Resolution Weight}

High resolution weight prefers clustering in sparse areas of the mesh and will therefore increase the precision of the visualization in very dense areas. This effect partially complement direction weight because high-curvature areas of triangle meshes are usually more dense for the high-curvature to be captured well.
\chapter{Implementation}

In-depth explanation of the solution. (Developer documentation.)

\begin{itemize}
\item Thoroughly describe the clustering techniques used
\item Thoroughly describe the new visualization methods
\item Mention how it is all put together in the application
\item Talk about loading/saving parameters/visualizations
\end{itemize}

\begin{algorithm}[H]
\caption{Clustering}
\begin{algorithmic}[1]

\Require ClusterSet c
\Statex
\For{all cells cell\textsubscript{i} \textbf{in} dataset}
	\State c = makeCluster(cell\textsubscript{i});
    \State set level of c to 0;
    \State add c to s;
\EndFor
\Statex
\For{all clusters c\textsubscript{i} in s}
	\For{all clusters c\textsubscript{j} neighbours of c\textsubscript{i}}
    	\State e = clusteringError(c\textsubscript{i}, c\textsubscript{j});
        \State insert pair (c\textsubscript{i}, c\textsubscript{j}) in increasing order of error e in a hash table;
        \State mark c\textsubscript{i} and c\textsubscript{j} as NOT\_CLUSTERED;
    \EndFor
\EndFor
\Statex
\State int i = 0;
\For{all pairs (c\textsubscript{i}, c\textsubscript{j}) in increasing order of error in the hash table}
	\If{both c\textsubscript{i} and c\textsubscript{j} are NOT\_CLUSTERED}
    	\State c = mergeClusters(c\textsubscript{i}, c\textsubscript{j});
        set level of c to l++;
        mark c\textsubscript{i} and c\textsubscript{j} as CLUSTERED;
        \For{all neighbors n\textsubscript{i} of c}
        	\State e = clusteringError(c, n\textsubscript{i});
            insert pair (c, n\textsubscript{i}) in order in hash table
        \EndFor
    \EndIf
\EndFor
\Statex
\Return c as root of tree
\end{algorithmic}
\end{algorithm}
\chapter{User Study}

In this chapter, we will concern ourselves with the user study we conducted, its setting, its results and the implications they bring.

\section{Setting}

The goal of our study was to simulate use cases of the visualizations and assess their performance in those use cases. We have prepared 10 questions, each of them tied to a specific pair of triangle meshes and the difference between them. For each question, we have prepared four different visualizations of the difference between the two meshes. Each participant had to answer each question with the help of one visualization randomly chosen out of the four. At the end, we have compared the answers received for each question among all four visualizations along with the time elapsed by the time the given participant confirmed their answer. In certain cases, we knew in advance what the correct answer to a given question was, in other cases the correct answer was unknown. Good visualizations therefore had to either help the participant answer correctly or create a clear agreement among the majority of participants.

\subsection{Data}

All triangle meshes used in the study were 3D scans of human faces kindly provided to us by the Laboratory of 3D Imaging and Analytical Methods of the Faculty of Science at Charles University. Overall, we have used six distinct mesh pairs to be able to study a larger range of difference and also to make the study more interesting to the participants.

\subsection{Visualizations}

For each question, we have chosen the four visualizations to be presented to the participants according to the following rules:

\begin{itemize}
\item Meshes without a visualization have to be included to allow for the contribution of visualizations in general to be measured
\item A visualization which we expected to be the most suitable for the given task according to out prediction outlined in section \ref{sec:analysis-color} has to be included
\item The remaining visualizations should be as distinct as possible in order to capture possible error in our predictions
\item At least one existing and one new visualization has to be included to allow for the differences between their performance to be captured
\item Each visualization type presented in this thesis has to be present at least once in the whole data set
\end{itemize}

After applying the visualizations to the six chosen mesh pairs, we have received 25 distinct pairs.

\subsection{Questions}

We have created the following types of questions based on the way one can answer them:

\begin{description}
\item [Left/Right] In these questions, participants had to answer by clicking on one of the meshes
\item [Yes/No] In these questions, participants had to answer by choosing either ``Yes'' or ``No'' from a dropdown menu
\item [Direction] In these questions, a specific direction selected from a dropdown menu was considered as an answer
\item [Location] Similarly, in these question, participants chose one of six predefined locations in a mesh in order to answer
\end{description}

Each question type also provided the possibility to answer ``Not sure'' when the participant did not understand the question or if the question was too difficult for them.

\subsection{Program}

We have modified MeshDiff for the purposes of the study. The only features which remained from MeshDiff were the ability to load and interactively view pre-generated visualizations and to modify the view by toggles described in appendix \ref{sec:view_control}. We have added a tutorial at the beginning which gives the full instructions to the participant (no prior knowledge is expected) and let them answer one sample question. 

We will now describe the course of one session of the study. At the beginning the participant receives full instructions (no prior knowledge of the program nor the subject of the study is required) and answers one sample question. The program then provides feedback to the answer and explains why it was correct or incorrect. Ten other questions are then presented, each begins with a description of the visualization currently being shown. After the participant agrees they have understood the description, the program starts to secretly measure time. When the answer is chosen and confirmed, time elapsed is saved along with the value of the provided answer and the next question is presented.

The study program runs locally without the ability to connect to the Internet and the participant has to manually upload the file containing their answers to a provided URL after they have finished.

\subsection{Participants}

Total of 36 volunteers of various backgrounds, ages and nationalities have participated in the study. Due to this number being relatively low, we were not able to analyze the answers of domain experts and the general public separately, nor were we able to make any other distinction. The primary intention, however, was to target the general public because we believe that this is aligned with the purpose of visualizations as a tool to make the understanding of data or concepts easier.

\section{Results}

We enclose the complete results of the study in attachment ... In this section, we will discuss the most significant results which are related to the limitations of existing visualization mentioned in the introduction or to the expected performance of new visualizations outlined in section \ref{sec:analysis-color}.

\subsection{The Overall Contribution of Visualizations}

\subsection{The Limitation of Color}

\subsection{The Strength of Thresholding}

\subsection{The Unexpected Benefit of Arrows}
\chapter{Discussion}

In this chapter, we will conclude the user study with a discussion of its results and suggest possible future improvements both to the study and to mesh difference visualizations in general.

%%-----------------------------------------------------------------------------------------
%% SECTION
%%-----------------------------------------------------------------------------------------
\section{Significant Results}

We will begin with an overview of the most significant results of the user study.

%%-----------------------------------------------------------------------------------------
\subsection{The Overall Contribution of Visualizations}

The answers to question 7 (see appendix \ref{sec:study-question7}), which requires participants to click on the face which has larger cheekbones, has shown an overwhelming dominance of visualizations over raw triangle meshes. Any of the three types of visualizations presented helped participants to answer correctly, whereas when no visualization was shown, answers were almost equally distributed among all the possible options. Moreover, when participants were shown meshes without a visualization, it took them longer to arrive at an answer. Similar effect could be observed in the answers to question 8 (see appendix \ref{sec:study-question8}).

%%-----------------------------------------------------------------------------------------
\subsection{The Strength of Thresholding}
\label{sec:discussion-thresholding}

In question 3 (see appendix \ref{sec:study-question3}), we tried to assess the performance of thresholding in a task which it was expected to excel at (see section \ref{sec:analysis_visualizations}), the visualization of the largest difference. We found that when the threshold was set just under the largest metric value, it was very easy for participants to identify the correct location. On the other hand, when a basic color visualization without thresholding was shown, the spread of answers was the largest and participants took the longest to answer. However, there are two interesting points associated with this question which we have to mention here:

\begin{itemize}
	\item This question clearly shows that participants were unsure what is meant by ``left'' and ``right'', even though this was explained at the beginning of the study.  
	\item Answers provided when no visualization was shown suggest that basic difference metrics mentioned in this thesis are insufficient in cases where the term ``difference'' is interpreted more globally. Humans are very sensitive to facial features and tend to see the differences between them more globally. This could also be observed in the answers to question 9 (see appendix \ref{sec:study-question9}).
\end{itemize}

%%-----------------------------------------------------------------------------------------
\subsection{The Unexpected Benefit of Arrows}

Question 9 (\ref{sec:study-question9}) has also shown a phenomenon which was not expected (see section \ref{sec:analysis_visualizations}). The question was aimed at the absolute value of the difference and where it was the smallest. Thresholding performed well as expected but most correct answers were given when arrows were shown. This shows that the length of an arrow is much more suggestive than the color scale when capturing the absolute value of a difference metric.

%%-----------------------------------------------------------------------------------------
%% SECTION
%%-----------------------------------------------------------------------------------------
\section{Study Improvements}

Because of time-constraints we were not able to improve the study and conduct it again in order to obtain a more complete and thorough comparison between color-based and arrow-based visualizations. A more detailed study would also be able to capture the differences between various clustering methods and clustering parameters which was beyond the resolving power of our study. In this section, we propose several improvements which might help to create such a study.

The most obvious limitation of our study was its scale. However, several other circumstances have further hindered our findings:

\begin{itemize}
	\item The overall presentation of the study did not attract enough attention. We recommend to design a web application with a modern and responsive interface which would allow for the sessions to be shorter, easier and more enjoyable to the participants. Being presented with a modern web application also adds to the feeling that the subject being studied is modern and relevant as well.
	\item A special attention should be given to the introduction of the study. Our results, such as \ref{sec:study-question3}, have shown that even though the way directions are given was explained in an introductory text, this text was too long and participants were not able to process it. This introduced error into the provided answers. We recommend to create an interactive introduction, rather than text-based.
	\item The last point related to user experience, which we believe is key, is the way questions are worded. Questions 4 (\ref{sec:study-question4}) and 5 (\ref{sec:study-question5}) have proven to be too complicated because of the amount of directions and concepts included in them. An example of a good question is question 7 (\ref{sec:study-question7}).
	\item Stepping away from user experience, we have noticed a problem related to the focus of the study. Our study has mixed questions related to visualizations (e.g. question 6 \ref{sec:study-question6}) with questions related to metrics (e.g. question 3 \ref{sec:study-question3}). On one hand, this allowed us to capture the need for more sophisticated difference metrics, on the other hand, it has not contributed to the comparison between various visualization types as strongly.
	\item Another problem related to this is the choice of data. As was mentioned in section \ref{sec:discussion-thresholding}, the difference metrics we have studied are not suitable for capturing the difference between human faces which people usually tend to see. While these metrics might still be useful in certain cases, this is not very clear from out study because of the data set chosen. We recommend to use more neutral objects in order to eliminate this phenomenon.
\end{itemize}

%%-----------------------------------------------------------------------------------------
%% SECTION
%%-----------------------------------------------------------------------------------------
\section{Method Improvements}

In order to improve the new visualizations presented in this thesis, a more detailed user study should be conducted as outlined above. However, during the course of our work we have discovered other possible improvements which were out of scope of this thesis. We present these here.

\begin{itemize}
	\item Firstly, the clustering process used in our visualization algorithm (see section \ref{sec:analysis_clustering_algorithm}), more specifically the dendrogram allows for an interactive visualization to be created. The user would be presented with the visualization of only one cluster covering the whole mesh. By clicking on a cluster, its children would be extracted instead from the dendrogram. This would result in a clustering of adaptive detail depending on the specific needs of the user.
	\item As mentioned in section \ref{sec:discussion-thresholding}, difference metrics included in this thesis are not sufficient in certain cases, such as when measuring the difference between facial features. A possible next step would be to devise global difference metrics which would better correspond to the idea of difference in shape that humans have. As a consequence, clustering would not be required because information would be reduced inherently.
\end{itemize}


\chapter*{Conclusion}
\addcontentsline{toc}{chapter}{Conclusion}

In this thesis, we have addressed the problem of the visualization of the difference between two triangle meshes. A common way of approaching this problem is to first compute a difference metric, for example the distance between corresponding vertices of the two meshes. This metric is then visualized. Existing visualizations mostly encode the metrics into mesh vertex color which may lead to information loss, especially when the metrics are multidimensional. We have proposed a new way of visualizing these metrics focusing on the ability to display multidimensional information and to group similar information together in order to prevent cluttering.

We have used the following two metrics (see Fig. \ref{fig:illustration-problem_input}):

\begin{itemize}
	\item Corresponding vertex distance	
	\item Corresponding vertex distance projected into surface normal
\end{itemize}

The proposed visualizations are the following (see section \ref{sec:analysis-visualizations}):

\begin{description}
	\item [Arrows] Clustered metric vectors are visualized using arrows.
	\item [Cluster color] Mesh vertices belonging to a given cluster are colored based on the properties of the cluster.
	\item [Thresholding] We have introduced a high-pass filter on the visualizations which excludes all information which is not significant enough.
	\item [Combined] Arrow-based and color-based visualizations can be combined to use the best of both worlds.
\end{description}

We have implemented these visualization in an experimental application called MeshDiff (see attachment \ref{attch:user_doc}) which we used to demonstrate the visualizations. The application can also be used more widely by people who want to create visualizations of their own data, export them, and present them in publications.

We have also conducted a user study which helped us evaluate the new visualizations as well as existing ones. It has shown that our visualizations are superior in certain areas, especially when comparing the absolute values of a metric and when underlining the most significant differences (see section \ref{sec:discussion-sig_results}).

\section*{Future Work}

Our study has also shown several deficiencies, both in its own design and in the visualizations. We have discovered three main areas of potential future improvement.

\begin{itemize}
	\item A better organized and more thorough study would be able to further assess the newly proposed and existing visualizations. Our study did not include the comparison of various clustering settings, for example. We believe these results would help to further improve these visualizations (see section \ref{sec:discussion-study_improvements}).
	\item Our study has suggested that existing difference metrics (which we have also used) are limited by their locality. Examining the differences between two human faces has manifested the need for more sophisticated global difference metrics (see section \ref{sec:discussion-method_improvements}).
	\item A new adaptive interactive visualization can be built on top of our clustering method where users would choose which areas they want to be visualized in more detail. This would further help to suppress unnecessary information (see section \ref{sec:discussion-method_improvements}).
\end{itemize}

%%% Bibliography
\include{bibliography}

%%% Figures used in the thesis (consider if this is needed)
\listoffigures

\begin{comment}

%%% Tables used in the thesis (consider if this is needed)
%%% In mathematical theses, it could be better to move the list of tables to the beginning of the thesis.
\listoftables

%%% Abbreviations used in the thesis, if any, including their explanation
%%% In mathematical theses, it could be better to move the list of abbreviations to the beginning of the thesis.
\chapwithtoc{List of Abbreviations}

%%% Attachments to the bachelor thesis, if any. Each attachment must be
%%% referred to at least once from the text of the thesis. Attachments
%%% are numbered.
%%%
%%% The printed version should preferably contain attachments, which can be
%%% read (additional tables and charts, supplementary text, examples of
%%% program output, etc.). The electronic version is more suited for attachments
%%% which will likely be used in an electronic form rather than read (program
%%% source code, data files, interactive charts, etc.). Electronic attachments
%%% should be uploaded to SIS and optionally also included in the thesis on a~CD/DVD.
%%% Allowed file formats are specified in provision of the rector no. 72/2017.
\appendix
\chapter{Attachments}

\section{First Attachment}

\end{comment}

\openright
\end{document}
